\documentclass[12pt,spanish]{article}
% aprovechamiento de la p\'agina -- fill an A4 (210mm x 297mm) page
% Note: 1 inch = 25.4 mm = 72.27 pt
% 1 pt = 3.5 mm (approx)

% vertical page layout -- one inch margin top and bottom
\topmargin      -10 mm   % top margin less 1 inch
\headheight       0 mm   % height of box containing the head
\headsep          0 mm   % space between the head and the body of the page
\textheight     255 mm   % the height of text on the page
\footskip         7 mm   % distance from bottom of body to bottom of foot

% horizontal page layout -- one inch margin each side
\oddsidemargin    0 mm     % inner margin less one inch on odd pages
\evensidemargin   0 mm     % inner margin less one inch on even pages
\textwidth      159 mm     % normal width of text on page

\setlength{\parindent}{0pt}

\usepackage{float}

\usepackage{amsmath}

\usepackage[doument]{ragged2e}
\usepackage{babel}
\usepackage[utf8]{inputenc}

% \usepackage{beton}
% \usepackage[T1]{fontenc}

\begin{document}

\title{SCD: Práctica4. Implementación de Sistemas de Tiempo Real}
\author{David Cabezas Berrido}
\date{\vspace{-5mm}}
\maketitle

\section*{Preguntas de la actividad 2}

Tareas a planificar:

\begin{table}[H]
  \centering
\begin{tabular}{|c|c|c|}
\hline
Tarea & $T$  & $C$ \\ \hline \hline
$A$   & 500  & 100 \\ \hline
$B$   & 500  & 150 \\ \hline
$C$   & 1000 & 200 \\ \hline
$D$   & 2000 & 240 \\ \hline
\end{tabular}
\end{table}

Hiperperíodo:

\[T_M=\text{mcm}(500,500,1000,2000) = 2000\]

Subperiodo:

\[T_s \geq 100=\text{min}(100,150,200,240)\]
\[T_s \leq 500=\text{min}(500,500,1000,2000) \qquad (\text{preferiblemente})\]
Tomo $T_s = 500$ 
\begin{verbatim}
 *---------*----------*---------*--------*
 | A B C   | A B D    | A B     | A B C  |
 *---------*----------*---------*--------*
 |         |          |         |        |
 0        500       1000      1500     2000
\end{verbatim}

\begin{itemize}
\item Tiempo mínimo de espera al final de las iteraciones del ciclo
  secundario: \\

  El tiempo es de 10 milisegundos, ocurre al final del segundo ciclo
  secundario.
  \[500-100-150-240=10\]

\item ¿Sería planificable si la tarea $D$ tuviese un tiempo de cómputo
  de 250ms? \\

  Teoricamente la misma planificación sería válida. Pero en la
  práctica, al mínimo retraso se incumple el deadline, porque el
  tiempo de cómputo del segundo ciclo secundario es justo el
  subperíodo.
\end{itemize}
\end{document}